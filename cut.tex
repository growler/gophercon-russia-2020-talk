% Presentation for my talk at GopherCon Russia 2020
%
% (c) Alexey Naidyonov 2020
% License CC-BY 3.0
%
\documentclass[
  11pt, aspectratio=1610,pdf,hyperref={unicode,colorlinks=false}
]{beamer}
\usetheme{gophercon}
\usepackage{pgfpages}
\usepackage[1080p]{itoolabs-graphics}
\usepackage{array}
\usepackage{tcolorbox}

\title{
  TLA+ Tools: \\
  практичный инструмент \\ 
  формальной верификации алгоритмов
}
\subtitle{
  (или что ещё разработчику на Golang точно\\надо изучить во время карантина)
}
\institute[\href{https://itoolabs.com}{itoolabs.com}]{%
  \href{https://itoolabs.com}{ITooLabs}}
\author[\href{https://twitter.com/growler}{@growler}]{%
  \href{https://twitter.com/growler}{Alexey Naidyonov}}
\event{\href{https://www.gophercon-russia.ru}{GopherCon Russia 2020}}
\date[28.03.2020]{Mar 28 2020}
\eventlogo{\includegraphics[keepaspectratio,width=.15\paperwidth]{gophercon-logo.png}}

\definecolor{shellbgcolor}{rgb}{1,0.99,0.96}

\begin{document}

\begingroup
\def\ident#1{\text{\small\itshape #1}}%
\def\str#1{\text{\small #1}}%
\def\cmd#1{\text{\small\sc #1}}%
\def\scmd#1{\textbackslash\relax#1}%
% Множества
\renewcommand*{\arraystretch}{1.2}
\begin{frame}[t,fragile]
  \centering{\Large\bf Sets (Множества)\\}\strut\\%
  \begin{tabular}{>{\begingroup\ttfamily\small}l<{\endgroup}@{\hspace{2ex}}>{\(}l<{\)}}
    Procs == \{1, 5, 8\}                       & \ident{Procs} \triangleq \{1, 5, 8\}\\
    Cardinality(Procs) = 3                     & \str{Cardinality}(\ident{Procs}) = 3\pause\\
    Ops == \{"inc", "dec"\}                    & \ident{Ops} \triangleq \{"\str{inc}", "\str{dec}"\}\pause\\
    Cmd == Ops \scmd{union} \{NULL\}           & \ident{Cmd} \triangleq \ident{Ops} \cup \{\ident{NULL}\}\\
    \{1, 2\} \scmd{intersect} \{2, 3\} = \{2\} & \{1, 2\} \cap \{2, 3\} = \{2\}\pause\\
    Var \scmd{in} 1..10                        & \ident{Var} \in 1..10\\
    ASSUME Var \scmd{notin} Procs              & \cmd{assume}\:\ident{Var} \notin \ident{Procs}\pause\\
    TRUE \scmd{in} BOOLEAN                     & \cmd{true} \in \cmd{boolean}\\
    "string" \scmd{in} STRING                  & ``\str{string}'' \in \cmd{string}\pause\\
    UNION \{\{1, 2\}, \{2, 3\}\} = \{1, 2, 3\} & \cmd{union}\:\{\{1, 2\}, \{2, 3\}\} = \{1, 2, 3\}
  \end{tabular}    
\end{frame}

\begin{frame}[t,fragile]
  \centering{\Large\bf Tuples (Кортежи)\\}\strut\\%
  \begin{tabular}{>{\ttfamily\small}l<{}@{\hspace{2ex}}>{\(}l<{\)}}
    a == <<1, 5, 8>>                           & \ident{a} \triangleq \langle 1, 5, 8\rangle \\
    a[2] = 5                                   & \ident{a}[2] = 5\pause\\
    Len(a) = 3                                 & \str{Len}(\ident{a}) = 3\\
    Head(a) = 1                                & \str{Head}(\ident{a}) = 1\\
    Tail(a) = <<5, 8>>                         & \str{Tail}(\ident{a}) = \langle 5, 8\rangle\\
    SubSeq(a, 1, 2) = <<1, 5>>                 & \str{SubSeq}(\ident{a}, 1, 2) = \langle 1, 5\rangle\pause\\
    <<1, 2>> \scmd{o} <<4, 8>> = <<1, 2, 4, 8>>& \langle 1, 2\rangle \circ \langle 4, 8\rangle = \langle 1, 2, 4, 8\rangle\\
    a := Append(a, 5)                          & \ident{a}' = \str{Append}(\ident{a}, 5)\pause\\
    a[i] := i                                  & \ident{a}' = [\ident{a}\:\cmd{except}\:![\ident{i}] = \ident{i}]\\
  \end{tabular}    
\end{frame}

\begin{frame}[t,fragile]
  \centering{\Large\bf Records (Записи)\\}\strut\\%
  \begin{tabular}{>{\ttfamily\small}l<{}@{\hspace{2ex}}>{\(}l<{\)}}
    s == [rdy|->0, ack|->0, val|->0]       & \ident{s} \triangleq [\str{rdy}\mapsto 0, \str{ack}\mapsto 0, \str{val}\mapsto 0]\\
    s.rdy = 0                              & \ident{s}.\str{rdy} = 0\\
    s["ack"] = 0                           & \ident{s}[``\str{ack}''] = 0\pause\\
    "a" :> 1 = [a|->1]                     & ``\str{a}'' :> 1 = [\str{a}\mapsto 1]\\\relax
    [a|->1] @@ [b|->2] = [a|->1, b|->2]    & [\str{a}\mapsto 1]\:@@\:[\str{b}\mapsto 2] = [\str{a}\mapsto 1, \str{b}\mapsto 2]\pause\\
    s.ack := 1                             & \ident{s}' = [\ident{s}\:\cmd{except}\:!.\str{ack} = 1]\\
    p.a := p.b || p.b = p.a                & \ident{p}' = [\ident{p}\:\cmd{except}\:!.\str{a} = \ident{p}.\str{b},\:!.\str{b} = \ident{p}.\str{a}]\\
  \end{tabular}    
\end{frame}

\begin{frame}[c,fragile]
  \centering{\Large\bf Functions (Функции)\\}\strut\\%
  \begin{tabular}{>{\ttfamily\small}l<{}@{\hspace{2ex}}>{\(}l<{\)}}
    Squares == [n \scmd{in} 1..Nat |-> n * n] & \ident{Squares} \triangleq [\ident{n} \in 1..\ident{Nat}\mapsto \ident{n}\cdot \ident{n}]\\
    Squares[2] = 4                         & \ident{Squares}[2] = 4\\
    DOMAIN Squares = 1..Nat                & \cmd{domain}\:\ident{Squares} = 1\dots\ident{Nat}\pause\\
    s == [rdy|->0, ack|->0, val|->0]       & \ident{s} \triangleq [\str{rdy}\mapsto 0, \str{ack}\mapsto 0, \str{val}\mapsto 0]\\
    DOMAIN s = \{"ack", "rdy", "val"\}     & \cmd{domain}\:\ident{s} = \{``\str{ack}'', ``\str{rdy}'', ``\str{val}''\}\pause\\
    DOMAIN <<5, 6, 7>> = 1..3              & \cmd{domain}\:\langle5, 6, 7\rangle = 1\dots 3\\         
  \end{tabular}
\end{frame}
\endgroup

\end{document}
%%% Local Variables:
%%% mode: latex
%%% TeX-master: t
%%% End:
